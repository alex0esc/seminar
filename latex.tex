\documentclass[a4paper,12pt,oneside]{scrartcl}  % legt die Art des Dokuments fest (benötigt *.sty-Datei mit entsprechendem Namen)

%%%%%%%%%%%%
% Packages % 
%%%%%%%%%%%%

% Umlaute und andere deutsche Besonderheiten
\usepackage[german,ngerman]{babel}  % damit die Überschriften "Inhaltsverzeichnis" etc. auf deutsch statt englisch erscheinen
\usepackage[utf8]{inputenc}  % damit man Umlaute direkt eingeben kann: ä statt \"a
\usepackage[T1]{fontenc}  % damit die Trennung auch bei Wörtern mit Umlauten funktioniert

\usepackage[autostyle=true,german=quotes]{csquotes}  % deutsche Anführungszeichen
\usepackage[german]{varioref}  % intelligente Verweise (z.B. "auf der nächsten Seite")

\usepackage[left=3cm,right=3cm,top=2cm,bottom=2.5cm]{geometry}  % Seitenmaße
\usepackage{setspace}  % zum Setzen des Zeilenabstands

% Schriftarten
\usepackage{palatino}  % alternative Schriftart
\usepackage{mathpazo}  % Palatino-Schriftart in Formeln

% Anhang
% \usepackage{appendix}

% Literaturverzeichnis
\usepackage{babelbib}  % damit das Literaturverzeichnis mehrsprachig sein kann
\usepackage[sort&compress,numbers]{natbib}  % für Literaturverzeichnis-Einstellmöglichkeiten - wird in den eckigen Klammern zusätzlich der Parameter "numbers" angegeben, dann ändert sich der Zitierstil von "Moore, 1965" zu "[1]" und die Quellen im Literaturverzeichnis werden entsprechend durchnummeriert

% Seitennummerierung variieren
\usepackage{scrlayer-scrpage}
\pagestyle{scrheadings}
\clearpairofpagestyles
\addtolength{\footskip}{-1.2cm}  % Seitenzahl etwas nach oben verschieben
\ofoot{\pagemark}  % Seitenzahl rechtsbündig in Fußzeile

% Einfügen von Bilddateien
\usepackage{graphicx}

% Mathematische Symbole und Umgebungen
\usepackage{amsmath}  % viele mathematische Umgebungen
\usepackage{amssymb}  % mathematische Symbole; lädt auch das Paket amsfonts nach
% \usepackage{amsthm}  % erweitert die Möglichkeiten der theorem-Umgebung
\usepackage{ziffer}  % damit hinter dem Dezimalkomma in Zahlen (z.B. 3,1415) kein Zwischenraum eingefügt wird; bei f(x, y) aber schon

\usepackage{siunitx}  % um Einheiten "gerade" und im passenden Abstand zur Maßzahl darzustellen - Kurzdokumentation hier: https://www.namsu.de/Extra/pakete/Siunitx.html
\sisetup{locale=DE, per-mode=symbol-or-fraction}

% Tabellen
\usepackage{array} 	% Verbesserung der array- und tabular-Umgebungen; sorgt für Formelnummerierung
\usepackage{tabularx}

% Nützliches
\usepackage[breaklinks, plainpages=false, bookmarks=true]{hyperref}  % Verweise mit Links versehen
% \usepackage{chngcntr}  % um Fußnoten über das gesamte Dokument zu zählen statt kapitelweise
\usepackage{eurosym}  % Euro-Symbol
\DeclareUnicodeCharacter{20AC}{\euro}  % damit man statt \euro einfach das €-Zeichen der Tastatur verwenden kann
\usepackage{blindtext}  % zum Erzeugen von Dummy-Blindtext, um mal schnell Seitenformatierungen und Textumfluss zu testen
\usepackage{verbatim}  % zum Einfügen von Programmcode o.ä. in Schreibmaschinenschrift (monospaced)



%%%%%%%%%%%%%%%%%%%%%%%%%%%%%%
% eigene Befehlsdefinitionen %
%%%%%%%%%%%%%%%%%%%%%%%%%%%%%%

% \newcommand{\R}{\mathbb{R}}  % Abkürzung \R für reelle Zahlen

%%%%%%%%%%%%%%%%%%%%%%%%%%%%%%%
% Einstellungen von Abständen %
%%%%%%%%%%%%%%%%%%%%%%%%%%%%%%%

% \setlength{\parindent}{0px}  % Einrückung am Absatzanfang
% \setlength{\parskip}{5pt}  % Abstand zum neuen Absatz (vertikal)
% \setlength{\belowcaptionskip}{-0.2em} % Abstand unter den Beschriftungen von Bildern und Tabellen
\renewcommand{\baselinestretch}{1.50}\normalsize % 1 1/2 Zeilig


%%%%%%%%%%%%%%%%%%%%%
%%%%%%%%%%%%%%%%%%%%%
%%% Hauptdokument %%%
%%%%%%%%%%%%%%%%%%%%%
%%%%%%%%%%%%%%%%%%%%%

\begin{document}
	
 \begin{titlepage}
  %----------------------------------------------------------------------
  %	Kopfzeile
  %----------------------------------------------------------------------

  \includegraphics[scale=0.2]{HGV-logo.png}\\[-2.2cm]
  \begin{flushright}
     \large Oberstufenjahrgang 2023/2025\\[1cm]
  \end{flushright}

  %----------------------------------------------------------------------
  %	Titel
  %----------------------------------------------------------------------

  \begin{center}
    \huge \bfseries Seminararbeit\\
    \Large \mdseries aus dem Fach Physik\\[1.5cm]
  \end{center}

  
  {\Large
  \begin{tabular}{lp{11.5cm}}
    Thema: & [Thema]
  \end{tabular}

  ~\\[0.5cm]

  %----------------------------------------------------------------------
  %	Autor, Bewertung
  %----------------------------------------------------------------------

  \large
  \begin{tabular}{ll}
    Verfasser: & Vorname Nachname \\
    W-Seminar: & Wissenschaftsjournalismus \\
    Seminarleiter: & Sebastian Bauer \\
    Abgabetermin: & xx. November 2024 \\[1cm]
    Erzielte Note: & \framebox[2cm][l]{\raisebox{0pt}[1.2em][0.1em]{~}}~~in Worten:~~\framebox[4.5cm][l]{\raisebox{0pt}[1.2em][0.1em]{~}} \\
    Erzielte Punkte: & \framebox[2cm][l]{\raisebox{0pt}[1.2em][0.1em]{~}}~~in Worten:~~\framebox[4.5cm][l]{\raisebox{0pt}[1.2em][0.1em]{~}} \\
    (einfache Wertung) & ~
  \end{tabular}
  
  \vspace{0.5cm}

  Abgabe beim Oberstufenbetreuer am:~~\framebox[5.6cm][l]{\raisebox{0pt}[1.2cm][0.3cm]{~}}
  
  \vspace{\fill}

  \begin{tabular}{cc} 
   \hspace{5cm} & \hspace{8.5cm} \\\cline{2-2} 
   ~ & Unterschrift des Seminarleiters 
  \end{tabular}
  }

 \end{titlepage}

 %----------------------------------------------------------------------
 %	Inhaltsverzeichnis
 %----------------------------------------------------------------------

 \begingroup
 \pagestyle{empty} % ohne Seitenzahl
 \tableofcontents % Inhaltsverzeichnis
 \clearpage % Seite bis zum Ende leeren
 \endgroup

 \newpage

 %----------------------------------------------------------------------
 %	erste Seite der Arbeit
 %----------------------------------------------------------------------

 \setcounter{page}{1} % Seitenzähler wird auf 1 gesetzt
	
 \section{Überschrift 1}\label{s1}
	
 \subsection{Abschnitt a}\label{s1ss1}
 
 \blindtext
 (nach \cite{Moore:1965}).
	
 \clearpage
	
 \subsection{Abschnitt b}\label{s1ss2}

 \blindtext
 \citep{FliessbachEDyn,DAMaike}
 \blindtext
 \citep{DissFWilhelm}

 \begin{equation}\label{eq1}
  D = \sqrt{b^2-4ac} 
 \end{equation}
 
 Die Gleichung (\vref{eq1}) kennst du sicher.\\[0.5cm]

 {\huge Text} siehe Abschnitt \vref{s1ss1}

 \clearpage
 
 \section{Überschrift 2}\label{s2}
	
 \blindtext
 
 \begin{table}[hb]
   \centering
   \begin{tabular}{r|c|l}
     Pos. & Zahl & Eigenschaften \\\hline
     1 & $\pi$ & eine runde Sache \\
     2 & $i$ & schon bisschen surreal \\
     3 & 42 & Keine Panik! \\
     4 & 23 & mal Sheldon fragen \\
   \end{tabular}\caption{Meine Lieblingszahlen}\label{tab:Lieblingszahlen}
 \end{table}
 
 \blindtext

 \begin{figure}[hbt]
   \centering
   \includegraphics[width=0.5\textwidth]{HGV-logo.png}
   \caption{Unser tolles Schullogo}\label{fig:HGVlogo}
 \end{figure}

 \blindtext
	
 \clearpage	
 \addcontentsline{toc}{section}{Abbildungsverzeichnis}
 \listoffigures

 \clearpage	
 \addcontentsline{toc}{section}{Tabellenverzeichnis}
 \listoftables

 \clearpage	
 \addcontentsline{toc}{section}{Literaturverzeichnis}
 %\bibliographystyle{ieeetr}
 \bibliographystyle{apalike-german}  % legt den Zitier- und Literaturverzeichnis-Stil fest (benötigt die *.bst-Datei mit entsprechendem Namen) 
 \bibliography{Vorlage}  % bindet die Datei Vorlage.bib als Literaturdatenbank ein
	
 \clearpage
 \appendix
 \addcontentsline{toc}{section}{Anhang}
 \section*{Anhang}
 %\subsection{Anhangsteil 1}
 %\subsection{Anhangsteil 2}
	
\end{document}
